\documentclass[preview]{standalone}

\usepackage[english]{babel}
\usepackage{amsmath}
\usepackage{amssymb}

\begin{document}

\begin{center}
To say that it is infinitely divided is no more than to say it actually has an infinite number of points at every one of which it is divisible. The point to note is that the infinite divisibility means not an infinite number of points of alternative division (such that the alternatives are inexhaustible) but rather an infinite number of points of simultaneous division. The points of division, being points on the being, belong to it not alternatively, but simultaneously. It is this simultaneity (and not a process) which is articulated by the postulate of the complete division. It is clear that if Zeno’s complete division thus is a cardinal completion rather than an ordinal completion, the infinite division of the given being does not imply a last division or last part, any more than the simultaneity of the points on a line imply an infinitieth point.
\end{center}

\end{document}
